%%%%%%%%%%%%%%%%%%%%%%%%%%%%%%%%%%%%%%%%%
% Structured General Purpose Assignment
% LaTeX Template
%
% This template has been downloaded from:
% http://www.latextemplates.com
%
% Original author:
% Ted Pavlic (http://www.tedpavlic.com)
%
% Note:
% The \lipsum[#] commands throughout this template generate dummy text
% to fill the template out. These commands should all be removed when 
% writing assignment content.
%
%%%%%%%%%%%%%%%%%%%%%%%%%%%%%%%%%%%%%%%%%

%----------------------------------------------------------------------------------------
%	PACKAGES AND OTHER DOCUMENT CONFIGURATIONS
%----------------------------------------------------------------------------------------

\documentclass{article}

\usepackage{fancyhdr} % Required for custom headers
\usepackage{lastpage} % Required to determine the last page for the footer
\usepackage{extramarks} % Required for headers and footers
\usepackage{graphicx} % Required to insert images
\usepackage{lipsum} % Used for inserting dummy 'Lorem ipsum' text into the template
\usepackage{enumerate}
\usepackage{booktabs}
\usepackage{amsmath}

% Margins
\topmargin=-0.45in
\evensidemargin=0in
\oddsidemargin=0in
\textwidth=6.5in
\textheight=9.0in
\headsep=0.25in 

\linespread{1.5} % Line spacing

% Set up the header and footer
\pagestyle{fancy}
\lhead{\hmwkAuthorName} % Top left header
\chead{\hmwkClass\ (\hmwkTitle)} % Top center header
%%\rhead{\firstxmark} 
\rhead{} % Top right header
\lfoot{\lastxmark} % Bottom left footer
\cfoot{} % Bottom center footer
\rfoot{Page\ \thepage\ of\ \pageref{LastPage}} % Bottom right footer
\renewcommand\headrulewidth{0.4pt} % Size of the header rule
\renewcommand\footrulewidth{0.4pt} % Size of the footer rule

\setlength\parindent{0pt} % Removes all indentation from paragraphs

%----------------------------------------------------------------------------------------
%	DOCUMENT STRUCTURE COMMANDS
%	Skip this unless you know what you're doing
%----------------------------------------------------------------------------------------

% Header and footer for when a page split occurs within a problem environment
\newcommand{\enterProblemHeader}[1]{
\nobreak\extramarks{#1}{#1 continued on next page\ldots}\nobreak
\nobreak\extramarks{#1 (continued)}{#1 continued on next page\ldots}\nobreak
}

% Header and footer for when a page split occurs between problem environments
\newcommand{\exitProblemHeader}[1]{
\nobreak\extramarks{#1 (continued)}{#1 continued on next page\ldots}\nobreak
\nobreak\extramarks{#1}{}\nobreak
}

\setcounter{secnumdepth}{0} % Removes default section numbers
\newcounter{homeworkProblemCounter} % Creates a counter to keep track of the number of problems

\newcommand{\homeworkProblemName}{}
\newenvironment{homeworkProblem}[1][Problem \arabic{homeworkProblemCounter}]{ % Makes a new environment called homeworkProblem which takes 1 argument (custom name) but the default is "Problem #"
\stepcounter{homeworkProblemCounter} % Increase counter for number of problems
\renewcommand{\homeworkProblemName}{#1} % Assign \homeworkProblemName the name of the problem
\section{\homeworkProblemName} % Make a section in the document with the custom problem count
\enterProblemHeader{\homeworkProblemName} % Header and footer within the environment
}{
\exitProblemHeader{\homeworkProblemName} % Header and footer after the environment
}

\newcommand{\problemAnswer}[1]{ % Defines the problem answer command with the content as the only argument
\noindent\framebox[\columnwidth][c]{\begin{minipage}{0.98\columnwidth}#1\end{minipage}} % Makes the box around the problem answer and puts the content inside
}

\newcommand{\homeworkSectionName}{}
\newenvironment{homeworkSection}[1]{ % New environment for sections within homework problems, takes 1 argument - the name of the section
\renewcommand{\homeworkSectionName}{#1} % Assign \homeworkSectionName to the name of the section from the environment argument
\subsection{\homeworkSectionName} % Make a subsection with the custom name of the subsection
\enterProblemHeader{\homeworkProblemName\ [\homeworkSectionName]} % Header and footer within the environment
}{
\enterProblemHeader{\homeworkProblemName} % Header and footer after the environment
}

%----------------------------------------------------------------------------------------
%	EXPECTATION AND VARIANCE OPERATOR
%----------------------------------------------------------------------------------------
 \newcommand{\E}{\mathrm{E}} 
 \newcommand{\Var}{\mathrm{Var}}
 \newcommand{\Cov}{\mathrm{Cov}}
 \newcommand{\Corr}{\mathrm{Corr}}
 
%----------------------------------------------------------------------------------------
%	NAME AND CLASS SECTION
%----------------------------------------------------------------------------------------

\newcommand{\hmwkTitle}{Homework\ \#1} % Assignment title
\newcommand{\hmwkDueDate}{Wednesday,\ February\ 7,\ 2018} % Due date
\newcommand{\hmwkClass}{FIN\ 591} % Course/class
\newcommand{\hmwkClassTime}{12:30pm} % Class/lecture time
\newcommand{\hmwkAuthorName}{Wanbae Park} % Your name

%----------------------------------------------------------------------------------------
%	TITLE PAGE
%----------------------------------------------------------------------------------------

\title{
\vspace{2in}
\textmd{\textbf{\hmwkClass:\ \hmwkTitle}}\\
\normalsize\vspace{0.1in}\small{Due\ on\ \hmwkDueDate}\\
\vspace{3in}
}

\author{\textbf{\hmwkAuthorName}}
\date{} % Insert date here if you want it to appear below your name

%----------------------------------------------------------------------------------------

\begin{document}

\maketitle

%----------------------------------------------------------------------------------------
%	TABLE OF CONTENTS
%----------------------------------------------------------------------------------------

%\setcounter{tocdepth}{1} % Uncomment this line if you don't want subsections listed in the ToC

%%\newpage
%%\tableofcontents
\newpage

%----------------------------------------------------------------------------------------
%	PROBLEM 1
%----------------------------------------------------------------------------------------
\begin{homeworkProblem}
	\begin{enumerate}[a.]	\label{prob(a)}
		%% Problem 1-a.
		\item \textit{(Short selling restriction)}
		If an investor cannot sell short risky asset, the amount of investment in risky asset should be nonnegative. Therefore, the maximization problem is stated as the following equation.
		\begin{equation} 	\label{eq:prob1-a: max 1}
		\begin{aligned}
			& \max_A \E[U(\widetilde{W})] = \max_A \E[U(W_0 (1 + r_f) + A(\widetilde{r} - r_f))] \\
			& \text{subject to } A \geq 0
		\end{aligned}
		\end{equation}
		The maximization problem (\ref{eq:prob1-a: max 1}) is equivalent to
%--------------------------------------------------------------------------------------------------------------
		%% Maximization Problem
		\begin{equation}	\label{eq:prob1-a: max 2}
			\max_A \E[U(W_0 (1 + r_f) + A(\widetilde{r} - r_f))] + \lambda A
		\end{equation}
		By applying Kuhn-Tucker conditions, the value of $A$ maximizing expected utility must must satisfy the following conditions.
%--------------------------------------------------------------------------------------------------------------
		%% First Order Conditions
		\begin{equation}	\label{eq:prob1-a: FOC}
		\begin{aligned}
			\E[U'(\widetilde{W})(\widetilde{r} - r_f)] + \lambda = 0		\\
			A \geq 0	\\
			A \E[U'(\widetilde{W})(\widetilde{r} - r_f)] = 0
		\end{aligned}
		\end{equation}
%--------------------------------------------------------------------------------------------------------------
%--------------------------------------------------------------------------------------------------------------
		%% Problem 1-b.
		\item \textit{(Restriction of riskless borrowing)}
		A restriction of riskless borrowing implies that the dollar amount of investment in riskless asset should be nonnegative. It leads to the following maximization problem.
%--------------------------------------------------------------------------------------------------------------
		%% Maximization Problem
		\begin{equation}	\label{eq:prob1-b: max 1}
		\begin{aligned}
			& \max_A \E[U(\widetilde{W})] = \max_A \E[U(W_0 (1 + r_f) + A(\widetilde{r} - r_f))] \\
			& \text{subject to } W_0 - A \geq 0			
		\end{aligned}
		\end{equation}
%--------------------------------------------------------------------------------------------------------------
		From the analogy of problem 1.a, the value of $A$ which maximizes expected utility must satisfy the following first order conditions.
		%% First Order Conditions
		\begin{equation}	\label{eq:prob1-b: FOC}
		\begin{aligned}
			\E[U'(\widetilde{W})(\widetilde{r} - r_f)] + \lambda = 0		\\
			W_0 - A \geq 0	\\
			(W_0 - A) \E[U'(\widetilde{W})(\widetilde{r} - r_f)] = 0
		\end{aligned}
		\end{equation}
	\end{enumerate}
\end{homeworkProblem}

%----------------------------------------------------------------------------------------
%	PROBLEM 2
%----------------------------------------------------------------------------------------
\begin{homeworkProblem}
	It is not necessary that two frontier portfolios which creates other efficient portfolio must be efficient. Although there is no risk-free asset, any portfolio can be generated by using market portfolio and zero beta portfolio. Since it is already known that zero beta portfolio is an inefficient portfolio, it is not necessary that two frontier portfolio must be efficient. Additionally, it can be explained intuitively. Let us consider a simple economy. In the economy, there are two assets, whose return and variance is denoted as $r_i$, $\sigma_i^2$, $i = 1, 2$, respectively. Let $\rho$ denote the correlation coefficient between $r_1$ and $r_2$. Now, construct a portfolio such that $r_p = w_1 r_1 + (1 - w_1) r_2$. In this case, the expected return of the portfolio is $w_1 \E[r_1] + (1 - w_2) \E[r_2]$, which is exactly on the straight line connecting $\E[r_1]$ to $\E[r_2]$ because it is just a weighted average of expected returns. However, since $\Var[r_p] = \Var[w_1 r_1 + (1 - w_1) r_2] = w_1^2 \sigma_1^2 + (1 - w_1)^2 \sigma_2^2 + \rho \sigma_1 \sigma_2$, variance of portfolio can be greater or less than weighted average of individual variances, depending on correlation coefficient, $\rho$. Therefore, if we properly choose portfolios which have negative correlation, we can create a portfolio which is on \underline{slightly left} to the straight line which connects $(\E[r_1], \sigma_1)$ and $(\E[r_2], \sigma_2)$ on $(\E[r], \sigma)$-space. (Actually, it justifies diversification effect.) It gives an intuition that we can create minimum variance portfolio(or similar portfolios) by appropriately choosing an efficient portfolio and an inefficient portfolio. Since minimum variance portfolio is also an efficient portfolio, it is possible to generate an efficient portfolio using two portfolios even if they are not both efficient.
\end{homeworkProblem}
%----------------------------------------------------------------------------------------
%	PROBLEM 3
%----------------------------------------------------------------------------------------
\begin{homeworkProblem}
	Return used to mean-variance analysis and CAPM should be nominal rate. It is because risk-free rate is adjusted(subtracted) in both mean-variance analysis and CAPM. Since risk-free rate already contain information of inflation(i.e. risk-free rate increases when inflation occurs, and decreases when deflation occurs), if we use real rate for mean-variance analysis or CAPM, inflation is "over-adjusted" in the analysis, and therefore it goes wrong. Hence, nominal returns should be used.
\end{homeworkProblem}
%----------------------------------------------------------------------------------------
%	PROBLEM 4
%----------------------------------------------------------------------------------------
\begin{homeworkProblem}
	\begin{enumerate}[a.]
		%% Problem 4-a.
		\item
		Let $x$ and $y$ denote weights of risky asset 1 and 2, respectively. Then since the expected return and variance of portfolio is $\overline{R}_p = x\overline{R}_1 + y\overline{R}_2 + (1 - x - y)R_f$, $\sigma^2_p = x^2 \sigma^2_1 + y^2 \sigma^2_2 +2xy\sigma_1 \sigma_2 \rho$, respectively, the Sharpe ratio of portfolio can be represented as follows.
%----------------------------------------------------------------------------------------
		%% Sharpe ratio representation
		\begin{equation}	\label{eq:prob4-a: Se representation}
		\begin{aligned}
			s_e 	&= \frac{\overline{R}_p - R_f}{\sigma_p}	\\
				&= \frac{x\overline{R}_1 + y\overline{R}_2 - (x + y)R_f}{\sqrt{x^2 \sigma^2_1 + y^2 \sigma^2_2 +2xy\sigma_1 \sigma_2 \rho}}	\\
				&= \frac{x \sigma_1 s_1 + y \sigma_2 s_2}{\sqrt{x^2 \sigma^2_1 + y^2 \sigma^2_2 +2xy\sigma_1 \sigma_2 \rho}}
		\end{aligned}
		\end{equation}
%----------------------------------------------------------------------------------------
		Let $A_1 = x \sigma_1 s_1 + y \sigma_2 s_2$ and $A_2 = (x^2 \sigma^2_1 + y^2 \sigma^2_2 +2xy\sigma_1 \sigma_2 \rho)^{-\frac{1}{2}}$. Then $s_e$ can be represented as $A_1 A_2$. By taking derivative to $s_e$ with respect to $x$ and $y$, we can obtain first order conditions as follows.
%----------------------------------------------------------------------------------------
		%% First order condition
		\begin{equation}	\label{eq:prob4-a: FOC1}
		\begin{aligned}
			\frac{\partial S_e}{\partial x} &= \frac{\partial A_1}{\partial x}A_2 + \frac{\partial A_2}{\partial x}A_1	\\
			\frac{\partial S_e}{\partial y} &= \frac{\partial A_1}{\partial y}A_2 + \frac{\partial A_2}{\partial y}A_1
		\end{aligned}
		\end{equation}
%----------------------------------------------------------------------------------------
		\begin{equation}	\label{eq:prob4-a: FOC2}
		\begin{aligned}
			\frac{\partial A_1}{\partial x} 	&= \sigma_1 s_1	\\
			\frac{\partial A_1}{\partial y} 	&= \sigma_2 s_2	\\
			\frac{\partial A_2}{\partial x} 	&= -\frac{1}{2}(x^2 \sigma^2_1 + y^2 \sigma^2_2 + 2xy \sigma_1 \sigma_2 \rho)^{-\frac{3}{2}} (2x \sigma^2_1 + 2y \sigma_1 \sigma_2 \rho)	\\
									&= -(x^2 \sigma^2_1 + y^2 \sigma^2_2 + 2xy \sigma_1 \sigma_2 \rho)^{-\frac{3}{2}} (x \sigma^2_1 + y \sigma_1 \sigma_2 \rho)	\\
			\frac{\partial A_2}{\partial y}	&= -\frac{1}{2}(x^2 \sigma^2_1 + y^2 \sigma^2_2 + 2xy \sigma_1 \sigma_2 \rho)^{-\frac{3}{2}} (2y \sigma^2_2 + 2x \sigma_1 \sigma_2 \rho)	\\
									&= -(x^2 \sigma^2_1 + y^2 \sigma^2_2 + 2xy \sigma_1 \sigma_2 \rho)^{-\frac{3}{2}} (y \sigma^2_2 + x \sigma_1 \sigma_2 \rho)
		\end{aligned}
		\end{equation}
%----------------------------------------------------------------------------------------
		Plugging the results of equation (\ref{eq:prob4-a: FOC2}) into equation (\ref{eq:prob4-a: FOC1}) we can get equations as follows.
%----------------------------------------------------------------------------------------	
		%% First order condition -- Se
		\begin{equation}	\label{eq:prob4-a: FOC-Se}
		\begin{aligned}
			\frac{\partial S_e}{\partial x} 	&= \sigma_1 s_1 (x^2 \sigma^2_1 + y^2 \sigma^2_2 +2xy\sigma_1 \sigma_2 \rho)^{-\frac{1}{2}} -(x^2 \sigma^2_1 + y^2 \sigma^2_2 + 2xy \sigma_1 \sigma_2 \rho)^{-\frac{3}{2}} (x \sigma^2_1 + y \sigma_1 \sigma_2 \rho) (x \sigma_1 s_1 + y \sigma_2 s_2)	\\
									&= (x^2 \sigma^2_1 + y^2 \sigma^2_2 +2xy\sigma_1 \sigma_2 \rho)^{-\frac{1}{2}} [\sigma_1 s_1 - (x^2 \sigma^2_1 + y^2 \sigma^2_2 + 2xy \sigma_1 \sigma_2 \rho)^{-1} (x \sigma^2_1 + y \sigma_1 \sigma_2 \rho) (x \sigma_1 s_1 + y \sigma_2 s_2)] \\ &= 0	\\
			\frac{\partial S_e}{\partial y} 	&= \sigma_2 s_2 (x^2 \sigma^2_1 + y^2 \sigma^2_2 +2xy\sigma_1 \sigma_2 \rho)^{-\frac{1}{2}} - (x^2 \sigma^2_1 + y^2 \sigma^2_2 + 2xy \sigma_1 \sigma_2 \rho)^{-\frac{3}{2}} (y \sigma^2_2 + x \sigma_1 \sigma_2 \rho) (x \sigma_1 s_1 + y \sigma_2 s_2)	\\
									&= (x^2 \sigma^2_1 + y^2 \sigma^2_2 +2xy\sigma_1 \sigma_2 \rho)^{-\frac{1}{2}} [\sigma_2 s_2 - (x^2 \sigma^2_1 + y^2 \sigma^2_2 + 2xy \sigma_1 \sigma_2 \rho)^{-1} (y \sigma^2_2 + x \sigma_1 \sigma_2 \rho) (x \sigma_1 s_1 + y \sigma_2 s_2)	\\ &= 0
		\end{aligned}
		\end{equation}	
%----------------------------------------------------------------------------------------	
		Since we assume that $x^2 \sigma^2_1 + y^2 \sigma^2_2 +2xy\sigma_1 \sigma_2 \rho \neq 0$, equation (\ref{eq:prob4-a: FOC-Se}) is reduced to equation (\ref{eq:prob4-a: FOC-Se2}).
%----------------------------------------------------------------------------------------	
		%% First order condition -- Se2
		\begin{equation}	\label{eq:prob4-a: FOC-Se2}
		\begin{aligned}
			\sigma_1 s_1 - (x^2 \sigma^2_1 + y^2 \sigma^2_2 + 2xy \sigma_1 \sigma_2 \rho)^{-1} (x \sigma^2_1 + y \sigma_1 \sigma_2 \rho) (x \sigma_1 s_1 + y \sigma_2 s_2) = 0	\\
			\sigma_2 s_2 - (x^2 \sigma^2_1 + y^2 \sigma^2_2 + 2xy \sigma_1 \sigma_2 \rho)^{-1} (y \sigma^2_2 + x \sigma_1 \sigma_2 \rho) (x \sigma_1 s_1 + y \sigma_2 s_2) = 0
		\end{aligned}
		\end{equation}
		Multiplying $x^2 \sigma^2_1 + y^2 \sigma^2_2 + 2xy \sigma_1 \sigma_2 \rho$ to both sides of equation (\ref{eq:prob4-a: FOC-Se2}), we obtain (\ref{eq:prob4-a: FOC-Se:x}) and (\ref{eq:prob4-a: FOC-Se:y}).
		%% First order condition -- Se3
		\begin{equation}
		\begin{aligned}		\label{eq:prob4-a: FOC-Se:x}
			&\sigma_1 s_1 (x^2 \sigma^2_1 + y^2 \sigma^2_2 + 2xy \sigma_1 \sigma_2 \rho) - (x \sigma^2_1 + y \sigma_1 \sigma_2 \rho) (x \sigma_1 s_1 + y \sigma_2 s_2) 	\\
				&= y^2 \sigma_1 \sigma^2_2 (s_1 - s_2 \rho) - xy \sigma^2_1 \sigma_2 (s_2 - s_1 \rho) = 0	\\
		\end{aligned}
		\end{equation}
		\begin{equation}
		\begin{aligned}		\label{eq:prob4-a: FOC-Se:y}
			&\sigma_2 s_2 (x^2 \sigma^2_1 + y^2 \sigma^2_2 + 2xy \sigma_1 \sigma_2 \rho) - (y \sigma^2_2 + x \sigma_1 \sigma_2 \rho) (x \sigma_1 s_1 + y \sigma_2 s_2) 	\\
				&= x^2 \sigma^2_1 \sigma_2 (s_2 - s_1 \rho) - xy \sigma_1 \sigma^2_2 (s_1 - s_2 \rho) = 0
		\end{aligned}
		\end{equation}
%----------------------------------------------------------------------------------------	
		By adding (\ref{eq:prob4-a: FOC-Se:x}) to (\ref{eq:prob4-a: FOC-Se:y}) we can obtain (\ref{eq:prob4-a: FOC-sum}).
		%% First order condition -- Sum
		\begin{equation}	\label{eq:prob4-a: FOC-sum}
		\begin{aligned}
			(x^2 & \sigma^2_1 \sigma_2 - xy \sigma^2_1 \sigma_2) (s_2 - s_1 \rho) + (y^2 \sigma_1 \sigma^2_2 - xy \sigma_1 \sigma^2_2) (s_1 - s_2 \rho) \\
			 	&= x \sigma^2_1 \sigma_2 (x - y) (s_2 - s_1 \rho) + y \sigma_1 \sigma^2_2 (y - x) (s_1 - s_2 \rho)	\\
				&= \sigma_1 \sigma_2 (x - y) [x \sigma_1 (s_2 - s_1 \rho) - y \sigma_2 (s_1 - s_2 \rho)] = 0
		\end{aligned}
		\end{equation}
%----------------------------------------------------------------------------------------	
		Assuming $x \neq y$, equation (\ref{eq:prob4-a: FOC-sum}) reduces to (\ref{eq:prob4-a: weight x}), which indicates optimal weight on risky asset 1.
%----------------------------------------------------------------------------------------	
		%% Optimal weight on risky asset 1
		\begin{equation}	\label{eq:prob4-a: weight x}
		\begin{aligned}
			x &\sigma_1 (s_2 - s_1 \rho) - y \sigma_2 (s_1 - s_2 \rho) = 0	\\
			&\Rightarrow x^* = \frac{y \sigma_2 (s_1 - s_2 \rho)}{\sigma_1 (s_2 - s_1 \rho)}
		\end{aligned}
		\end{equation}
%----------------------------------------------------------------------------------------	
		Plug $x^*$ into equation (\ref{eq:prob4-a: Se representation}), we can get the maximum Sharpe ratio as follows.
		%% Maximized Sharpe raio
		\begin{equation}	\label{eq:prob4-a: max sharpe}
		\begin{aligned}
			&s_e^*	 = \frac{x^* \sigma_1 s_1 + y \sigma_2 s_2}{\sqrt{x^{*2} \sigma^2_1 + y^2 \sigma^2_2 +2x^*y\sigma_1 \sigma_2 \rho}}	\\
			&x^* \sigma_1 s_1 + y \sigma_2 s_2	= \frac{y \sigma_2 (s_1 - s_2 \rho)}{\sigma_1 (s_2 - s_1 \rho)} \sigma_1 s_1 + y \sigma_2 s_2		\\
										&= y \sigma_2 \frac{s_1 - s_2 \rho}{s_2 - s_1 \rho} + s_2 ~~ \text(\textit{numerator})	\\
			& \sqrt{x^{*2} \sigma^2_1 + y^2 \sigma^2_2 +2x^*y\sigma_1 \sigma_2 \rho} = y \sigma_2 \sqrt{(\frac{s_1 - s_2 \rho}{s_2 - s_1 \rho})^2 + 2 \frac{s_1 - s_2 \rho}{s_2 - s_1 \rho} + 1} ~~ \text(\textit{denominator})	\\
			& \Rightarrow s_e^* = \frac{s_1^2 - 2s_1 s_2 \rho + s_2^2}{[(s_1 - s_2 \rho)^2 + 2(s_1 - s_2 \rho)(s_2 - s_1 \rho)\rho + (s_2 - s_1 \rho)^2]^{\frac{1}{2}}}
		\end{aligned}
		\end{equation}
%----------------------------------------------------------------------------------------	
		%% Problem 4-b.
		\item
		From 4.a, we derived that maximum Sharpe ratio is attained if $x = \frac{y \sigma_2 (s_1 - s_2 \rho)}{\sigma_1 (s_2 - s_1 \rho)}$. Therefore, since $\sigma_1$ and $\sigma_2$ are both positive, in order to achieve maximum Sharpe ratio under constraints that $x > 0$ and $y > 0$, it is necessary that $\frac{s_1 - s_2 \rho}{s_2 - s_1 \rho}$ should be positive.
		\end{enumerate}
\end{homeworkProblem}
\end{document}