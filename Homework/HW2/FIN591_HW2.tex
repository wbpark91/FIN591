%%%%%%%%%%%%%%%%%%%%%%%%%%%%%%%%%%%%%%%%%
% Structured General Purpose Assignment
% LaTeX Template
%
% This template has been downloaded from:
% http://www.latextemplates.com
%
% Original author:
% Ted Pavlic (http://www.tedpavlic.com)
%
% Note:
% The \lipsum[#] commands throughout this template generate dummy text
% to fill the template out. These commands should all be removed when 
% writing assignment content.
%
%%%%%%%%%%%%%%%%%%%%%%%%%%%%%%%%%%%%%%%%%

%----------------------------------------------------------------------------------------
%	PACKAGES AND OTHER DOCUMENT CONFIGURATIONS
%----------------------------------------------------------------------------------------

\documentclass{article}

\usepackage{fancyhdr} % Required for custom headers
\usepackage{lastpage} % Required to determine the last page for the footer
\usepackage{extramarks} % Required for headers and footers
\usepackage{graphicx} % Required to insert images
\usepackage{lipsum} % Used for inserting dummy 'Lorem ipsum' text into the template
\usepackage{enumerate}
\usepackage{booktabs}
\usepackage{amsmath}

% Margins
\topmargin=-0.45in
\evensidemargin=0in
\oddsidemargin=0in
\textwidth=6.5in
\textheight=9.0in
\headsep=0.25in 

\linespread{1.5} % Line spacing

% Set up the header and footer
\pagestyle{fancy}
\lhead{\hmwkAuthorName} % Top left header
\chead{\hmwkClass\ (\hmwkTitle)} % Top center header
%%\rhead{\firstxmark} 
\rhead{} % Top right header
\lfoot{\lastxmark} % Bottom left footer
\cfoot{} % Bottom center footer
\rfoot{Page\ \thepage\ of\ \pageref{LastPage}} % Bottom right footer
\renewcommand\headrulewidth{0.4pt} % Size of the header rule
\renewcommand\footrulewidth{0.4pt} % Size of the footer rule

\setlength\parindent{0pt} % Removes all indentation from paragraphs

%----------------------------------------------------------------------------------------
%	DOCUMENT STRUCTURE COMMANDS
%	Skip this unless you know what you're doing
%----------------------------------------------------------------------------------------

% Header and footer for when a page split occurs within a problem environment
\newcommand{\enterProblemHeader}[1]{
\nobreak\extramarks{#1}{#1 continued on next page\ldots}\nobreak
\nobreak\extramarks{#1 (continued)}{#1 continued on next page\ldots}\nobreak
}

% Header and footer for when a page split occurs between problem environments
\newcommand{\exitProblemHeader}[1]{
\nobreak\extramarks{#1 (continued)}{#1 continued on next page\ldots}\nobreak
\nobreak\extramarks{#1}{}\nobreak
}

\setcounter{secnumdepth}{0} % Removes default section numbers
\newcounter{homeworkProblemCounter} % Creates a counter to keep track of the number of problems

\newcommand{\homeworkProblemName}{}
\newenvironment{homeworkProblem}[1][Problem \arabic{homeworkProblemCounter}]{ % Makes a new environment called homeworkProblem which takes 1 argument (custom name) but the default is "Problem #"
\stepcounter{homeworkProblemCounter} % Increase counter for number of problems
\renewcommand{\homeworkProblemName}{#1} % Assign \homeworkProblemName the name of the problem
\section{\homeworkProblemName} % Make a section in the document with the custom problem count
\enterProblemHeader{\homeworkProblemName} % Header and footer within the environment
}{
\exitProblemHeader{\homeworkProblemName} % Header and footer after the environment
}

\newcommand{\problemAnswer}[1]{ % Defines the problem answer command with the content as the only argument
\noindent\framebox[\columnwidth][c]{\begin{minipage}{0.98\columnwidth}#1\end{minipage}} % Makes the box around the problem answer and puts the content inside
}

\newcommand{\homeworkSectionName}{}
\newenvironment{homeworkSection}[1]{ % New environment for sections within homework problems, takes 1 argument - the name of the section
\renewcommand{\homeworkSectionName}{#1} % Assign \homeworkSectionName to the name of the section from the environment argument
\subsection{\homeworkSectionName} % Make a subsection with the custom name of the subsection
\enterProblemHeader{\homeworkProblemName\ [\homeworkSectionName]} % Header and footer within the environment
}{
\enterProblemHeader{\homeworkProblemName} % Header and footer after the environment
}

%----------------------------------------------------------------------------------------
%	EXPECTATION AND VARIANCE OPERATOR
%----------------------------------------------------------------------------------------
 \newcommand{\E}{\mathrm{E}} 
 \newcommand{\Var}{\mathrm{Var}}
 \newcommand{\Cov}{\mathrm{Cov}}
 \newcommand{\Corr}{\mathrm{Corr}}
 
%----------------------------------------------------------------------------------------
%	NAME AND CLASS SECTION
%----------------------------------------------------------------------------------------

\newcommand{\hmwkTitle}{Homework\ \#2} % Assignment title
\newcommand{\hmwkDueDate}{Wednesday,\ February\ 28,\ 2018} % Due date
\newcommand{\hmwkClass}{FIN\ 591} % Course/class
\newcommand{\hmwkClassTime}{12:30pm} % Class/lecture time
\newcommand{\hmwkAuthorName}{Wanbae Park} % Your name

%----------------------------------------------------------------------------------------
%	TITLE PAGE
%----------------------------------------------------------------------------------------

\title{
\vspace{2in}
\textmd{\textbf{\hmwkClass:\ \hmwkTitle}}\\
\normalsize\vspace{0.1in}\small{Due\ on\ \hmwkDueDate}\\
\vspace{3in}
}

\author{\textbf{\hmwkAuthorName}}
\date{} % Insert date here if you want it to appear below your name

%----------------------------------------------------------------------------------------

\begin{document}

\maketitle

%----------------------------------------------------------------------------------------
%	TABLE OF CONTENTS
%----------------------------------------------------------------------------------------

%\setcounter{tocdepth}{1} % Uncomment this line if you don't want subsections listed in the ToC

%%\newpage
%%\tableofcontents
\newpage

%----------------------------------------------------------------------------------------
%	PROBLEM 1
%----------------------------------------------------------------------------------------
\begin{homeworkProblem}
\begin{enumerate}[a.]
	\item		%% Problem 1.a.
	\item		%% Problem 1.b.
\end{enumerate}
\end{homeworkProblem}

%----------------------------------------------------------------------------------------
%	PROBLEM 2
%----------------------------------------------------------------------------------------
\begin{homeworkProblem}
\begin{enumerate}[a.]
	\item		%% Problem 2.a.
	Under the optimal choice, $U_C(C_{T - 1}, T - 1) = \E_{T - 1}[B_W(W_T, T) R_{T - 1}]$ holds. If we plug given utility and bequest function to the equation, the following equation holds.
%----------------------------------------------------------------------------------------
	%% FOC - 1
	\begin{equation} \label{eq:prob2-a: FOC1}
	\begin{aligned}
		&\delta^{T - 1} C_{T - 1}^{\gamma - 1} = \E_{T - 1}[\delta^T W_{T}^{\gamma - 1} R_{T - 1}]	\\
		\Rightarrow \delta^{T - 1} C_{T - 1}^{\gamma - 1} &= \delta^T S_{T - 1}^{\gamma - 1} \E[R_{T - 1}^\gamma] ~~ \text{where~} W_T = S_{T - 1}R_{T - 1}, S_{T - 1} = W_{T - 1} - C_{T - 1}
	\end{aligned}
	\end{equation}
%----------------------------------------------------------------------------------------
	Therefore, if we rearrange the equation (\ref{eq:prob2-a: FOC1}), the optimal consumption at time $T - 1$, $C_{T - 1}^*$ can be obtained as follows.
%----------------------------------------------------------------------------------------
	%% FOC - 2
	\begin{equation}	\label{eq:prob2-a: FOC2}
		C_{T - 1}^* = \frac{ \delta^{ \frac{1}{\gamma - 1} } \E_{T - 1}[R_{T - 1}^{\gamma}]^{ \frac{1}{ \gamma - 1 }}}{1 - \delta^{ \frac{1}{\gamma - 1} } \E_{T - 1}[R_{T - 1}^{\gamma}]^{\frac{1}{\gamma - 1}}}W_{T - 1}
	\end{equation}
	Another condition under optimal choice is $\E_{T - 1}[B_W(W_T, T) (R_{i, T - 1} - R_f)] = 0$ for $i = 1, 2, 3, \dots , n$. Therefore, the following equation holds.
%----------------------------------------------------------------------------------------
	%% FOC - 3
	\begin{equation}	\label{eq:prob2-a: FOC3}
	\begin{aligned}
		&\E_{T - 1}[\delta^T W_T^{\gamma - 1} R_{i, T - 1}] = R_f \E_{T - 1} [\delta^T W_T^{\gamma - 1}]		\\
		&\Rightarrow \E_{T - 1}[(S_{T - 1} R_{T - 1})^{\gamma - 1} R_{i, T - 1}] = R_f \E_{T - 1} [\delta^T (S_{T - 1} R_{T - 1})^{\gamma -1}]	\\
		&\Rightarrow \E_{T - 1}[R_{T - 1}^{\gamma - 1} R_{i, T - 1}] = R_f \E_{T - 1}[R_{T - 1}^{\gamma - 1}]
	\end{aligned}
	\end{equation}
%----------------------------------------------------------------------------------------
	\item		%% Problem 2.b.
	Let $\delta^{ \frac{1}{\gamma - 1} } \E_{T - 1}[R_{T - 1}^{\gamma}]^{ \frac{1}{ \gamma - 1 }} = a$. Then $C_{T - 1}^* = \frac{a}{1 + a} W_{T - 1}$. Since $J(W_{T - 1}, T - 1) = U(C_{T - 1}^*, T - 1) = \E_{T - 1}[B(W_T, T)]$, $J(W_{T - 1}, T - 1)$ can be represented as follows.
%----------------------------------------------------------------------------------------
	%% J Representation
	\begin{equation}	\label{eq:prob2-b: J representation}
	\begin{aligned}
		J(W_{T - 1}, T - 1) 	&= \frac{\delta^{T - 1} C_{T - 1}^{*\gamma}}{\gamma} + \E_{T - 1}[\frac{\delta^T W_T^{\gamma}}{\gamma}]	\\
						&= \frac{\delta^{T - 1}}{\gamma} (\frac{a}{1 + a}W_{T - 1})^{\gamma} + \frac{\delta^T}{\gamma} \E_{T - 1}[((1 - \frac{a}{1 + a})W_{T - 1}R_{T - 1})^{\gamma}]	\\
						& = \frac{\delta^{T - 1}}{\gamma} ((\frac{a}{1 + a})^{\gamma} W_{T - 1}^{\gamma} + \delta (\frac{1}{1 + a})^{\gamma} W_{T - 1}^{\gamma} \E_{T - 1}[R_{T - 1}^{\gamma}])	\\
						&= \frac{\delta^{T - 1}}{\gamma} (\frac{1}{1 + a})^{\gamma} (a^\gamma + \delta \E_{T - 1}[R_{T - 1}^\gamma]) W_{T - 1}^\gamma
	\end{aligned}
	\end{equation}
	\item 	%% Problem 2.c.
	Let $\frac{1}{1 + a}^\gamma (a^\gamma + \delta \E_{T - 1}[R_{T - 1}^\gamma])= b$. Then $J(W_{T - 1}, T - 1)$ can be represented as $\frac{\delta^{T - 1}}{\gamma} b W_{T - 1}^\gamma$. Since under optimal choice, $U_C(C_{T - 2}, T - 2) = \E_{T - 2}[J_W(W_{T - 1}, T - 1)R_{T - 2}]$ holds, the following equation must hold.
%----------------------------------------------------------------------------------------
	%%FOC - 1
	\begin{equation}	\label{eq:prob2-c: FOC1}
	\begin{aligned}
		\delta^{T - 2} C_{T - 2}^{\gamma -1} 	&= \E_{T - 2}[\delta^{T - 1} b W_{T - 1}^{\gamma - 1} R_{T - 2}]	\\
									&= \delta^{T - 1} b \E_{T - 2}[S_{T - 2}^{\gamma - 1} R_{T - 2}^\gamma]	\\
									&= \delta^{T - 1} b (W_{T - 2} - C_{T - 2})^{\gamma - 1} \E_{T - 2}[R_{T - 2}^\gamma]		\\
		C_{T - 2}						&= \delta b (W_{T - 2} - C_{T - 2}) \E_{T - 2}[R_{T - 2}]^{\frac{1}{\gamma -1}}
	\end{aligned}
	\end{equation}
%----------------------------------------------------------------------------------------
	By rearranging the terms in equation (\ref{eq:prob2-c: FOC1}), we can get an explicit form of $C_{T - 2}^*$ as follows.
%----------------------------------------------------------------------------------------
	%%FOC - 2
	\begin{equation}	\label{eq:prob2-c: FOC2}
	\begin{aligned}
		C_{T - 2}^* 	&= \frac{ \delta b \E_{T - 2}[R_{T - 2}^\gamma]^{\frac{1}{\gamma - 1}} }{1 + \delta b \E_{T - 2}[R_{T - 2}^\gamma]^{\frac{1}{\gamma - 1}} } W_{T - 2}	\\
					&= \frac{c}{1 + c} W_{T - 2} ~~ \text{where}~ c = \delta b \E_{T - 2}[R_{T - 2}^\gamma]^{\frac{1}{\gamma - 1}}
	\end{aligned}
	\end{equation}
	Another optimal condition is $\E_{T - 2}[R_{i, T - 2} J_W(W_{T - 1}, T - 1)] = R_f \E_{T - 2}[J_W(W_{T - 1}, T - 1)]$. Therefore, the following equations hold.%----------------------------------------------------------------------------------------
	%%FOC - 3
	\begin{equation}
	\begin{aligned}
		&\E_{T - 2}[R_{i, T - 2} \delta^{T - 1} b W_{T - 1}^{\gamma - 1}] = R_f \E_{T - 2}[\delta^{T - 1} b W_{T - 1}^{\gamma - 1}]	\\
		&\Rightarrow \E_{T - 2}[R_{i, T - 2} W_{T - 1}^{\gamma - 1}] = R_f \E_{T - 2} [W_{T - 1}^{\gamma - 1}]	\\
		&\Rightarrow \E_{T - 2}[R_{i, T - 2} (S_{T - 2} R_{T - 2})^{\gamma - 1}] = R_f \E_{T - 2} [(S_{T - 2} R_{T - 2})^{\gamma - 1}]	\\
		&\Rightarrow \E_{T - 2}[R_{i, T - 2} R_{T - 2}^{\gamma - 1}] = R_f \E_{T - 2} [R_{T - 2}^{\gamma - 1}]
	\end{aligned}
	\end{equation}
	
%----------------------------------------------------------------------------------------
	
	\item 	%% Problem 2.d.
\end{enumerate}
\end{homeworkProblem}
\end{document}