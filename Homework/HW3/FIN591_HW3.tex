%%%%%%%%%%%%%%%%%%%%%%%%%%%%%%%%%%%%%%%%%
% Structured General Purpose Assignment
% LaTeX Template
%
% This template has been downloaded from:
% http://www.latextemplates.com
%
% Original author:
% Ted Pavlic (http://www.tedpavlic.com)
%
% Note:
% The \lipsum[#] commands throughout this template generate dummy text
% to fill the template out. These commands should all be removed when
% writing assignment content.
%
%%%%%%%%%%%%%%%%%%%%%%%%%%%%%%%%%%%%%%%%%

%----------------------------------------------------------------------------------------
%	PACKAGES AND OTHER DOCUMENT CONFIGURATIONS
%----------------------------------------------------------------------------------------

\documentclass{article}

\usepackage{fancyhdr} % Required for custom headers
\usepackage{lastpage} % Required to determine the last page for the footer
\usepackage{extramarks} % Required for headers and footers
\usepackage{graphicx} % Required to insert images
\usepackage{lipsum} % Used for inserting dummy 'Lorem ipsum' text into the template
\usepackage{enumerate}
\usepackage{booktabs}
\usepackage{amsmath}

% Margins
\topmargin=-0.45in
\evensidemargin=0in
\oddsidemargin=0in
\textwidth=6.5in
\textheight=9.0in
\headsep=0.25in

\linespread{1.5} % Line spacing

% Set up the header and footer
\pagestyle{fancy}
\lhead{\hmwkAuthorName} % Top left header
\chead{\hmwkClass\ (\hmwkTitle)} % Top center header
%%\rhead{\firstxmark}
\rhead{} % Top right header
\lfoot{\lastxmark} % Bottom left footer
\cfoot{} % Bottom center footer
\rfoot{Page\ \thepage\ of\ \pageref{LastPage}} % Bottom right footer
\renewcommand\headrulewidth{0.4pt} % Size of the header rule
\renewcommand\footrulewidth{0.4pt} % Size of the footer rule

\setlength\parindent{0pt} % Removes all indentation from paragraphs

%----------------------------------------------------------------------------------------
%	DOCUMENT STRUCTURE COMMANDS
%	Skip this unless you know what you're doing
%----------------------------------------------------------------------------------------

% Header and footer for when a page split occurs within a problem environment
\newcommand{\enterProblemHeader}[1]{
\nobreak\extramarks{#1}{#1 continued on next page\ldots}\nobreak
\nobreak\extramarks{#1 (continued)}{#1 continued on next page\ldots}\nobreak
}

% Header and footer for when a page split occurs between problem environments
\newcommand{\exitProblemHeader}[1]{
\nobreak\extramarks{#1 (continued)}{#1 continued on next page\ldots}\nobreak
\nobreak\extramarks{#1}{}\nobreak
}

\setcounter{secnumdepth}{0} % Removes default section numbers
\newcounter{homeworkProblemCounter} % Creates a counter to keep track of the number of problems

\newcommand{\homeworkProblemName}{}
\newenvironment{homeworkProblem}[1][Problem \arabic{homeworkProblemCounter}]{ % Makes a new environment called homeworkProblem which takes 1 argument (custom name) but the default is "Problem #"
\stepcounter{homeworkProblemCounter} % Increase counter for number of problems
\renewcommand{\homeworkProblemName}{#1} % Assign \homeworkProblemName the name of the problem
\section{\homeworkProblemName} % Make a section in the document with the custom problem count
\enterProblemHeader{\homeworkProblemName} % Header and footer within the environment
}{
\exitProblemHeader{\homeworkProblemName} % Header and footer after the environment
}

\newcommand{\problemAnswer}[1]{ % Defines the problem answer command with the content as the only argument
\noindent\framebox[\columnwidth][c]{\begin{minipage}{0.98\columnwidth}#1\end{minipage}} % Makes the box around the problem answer and puts the content inside
}

\newcommand{\homeworkSectionName}{}
\newenvironment{homeworkSection}[1]{ % New environment for sections within homework problems, takes 1 argument - the name of the section
\renewcommand{\homeworkSectionName}{#1} % Assign \homeworkSectionName to the name of the section from the environment argument
\subsection{\homeworkSectionName} % Make a subsection with the custom name of the subsection
\enterProblemHeader{\homeworkProblemName\ [\homeworkSectionName]} % Header and footer within the environment
}{
\enterProblemHeader{\homeworkProblemName} % Header and footer after the environment
}

%----------------------------------------------------------------------------------------
%	EXPECTATION AND VARIANCE OPERATOR
%----------------------------------------------------------------------------------------
 \newcommand{\E}{\mathrm{E}}
 \newcommand{\Var}{\mathrm{Var}}
 \newcommand{\Cov}{\mathrm{Cov}}
 \newcommand{\Corr}{\mathrm{Corr}}

%----------------------------------------------------------------------------------------
%	NAME AND CLASS SECTION
%----------------------------------------------------------------------------------------

\newcommand{\hmwkTitle}{Homework\ \#3} % Assignment title
\newcommand{\hmwkDueDate}{Wednesday,\ April\ 11,\ 2018} % Due date
\newcommand{\hmwkClass}{FIN\ 591} % Course/class
\newcommand{\hmwkClassTime}{12:30pm} % Class/lecture time
\newcommand{\hmwkAuthorName}{Wanbae Park} % Your name

%----------------------------------------------------------------------------------------
%	TITLE PAGE
%----------------------------------------------------------------------------------------

\title{
\vspace{2in}
\textmd{\textbf{\hmwkClass:\ \hmwkTitle}}\\
\normalsize\vspace{0.1in}\small{Due\ on\ \hmwkDueDate}\\
\vspace{3in}
}

\author{\textbf{\hmwkAuthorName}}
\date{} % Insert date here if you want it to appear below your name

%----------------------------------------------------------------------------------------

\begin{document}

\maketitle

%----------------------------------------------------------------------------------------
%	TABLE OF CONTENTS
%----------------------------------------------------------------------------------------

%\setcounter{tocdepth}{1} % Uncomment this line if you don't want subsections listed in the ToC

%%\newpage
%%\tableofcontents
\newpage

%----------------------------------------------------------------------------------------
%	PROBLEM 1
%----------------------------------------------------------------------------------------
\begin{homeworkProblem}
    \begin{enumerate}[a.]
        \item   %% a.
        Since the final payoff of $P$ is 1, using continuous-time version
        stochastic discount factor, $P_t(\tau)$ is derived as follows.
        \begin{equation}
            \begin{aligned}
                P_t(\tau)   &= \E_t\left[ \frac{U_c(C_{t + \tau, t + \tau})}{U_c(C_t, t)} \times 1 \right]    \\
                            &= \E_t\left[ \frac{e^{-\phi(t + \tau)} C_{t + \tau}^{\gamma - 1}}{e^{\phi t} C_t^{\gamma - 1}} \right] \\
                            &= \E_t \left[ e^{-\phi \tau} \frac{C_{t + \tau}^{\gamma - 1}}{C_t^{\gamma - 1}} \right]
            \end{aligned}
            \label{eq:prob1}
        \end{equation}
        \item   %% b.
        From $P_t(\tau) = \E_t\left[ \frac{e^{-\phi(t + \tau)} C_{t + \tau}^{\gamma - 1}}{e^{\phi t} C_t^{\gamma - 1}} \right]$,
        we can find that process $M_t$ is equal to $e^{-\phi t}C_t^{\gamma - 1}$.
        Therefore, using Ito's lemma, dynamics of $M_t$ can be derived as
        equation (\ref{eq:prob1-b}).
        \begin{equation}
            \begin{aligned}
                dM_t    &= -\phi e^{-\phi t} C^{\gamma - 1}dt
                            + e^{-\phi t} (\gamma - 1) C^{\gamma - 2}
                            C [(\mu_c - \lambda k)dt + \sigma_c dZ_c]   \\
                            &+ \frac{1}{2} e^{-\phi t} (\gamma - 1)(\gamma - 2)
                            C^2 C^{\gamma - 3} \sigma_c^2 dt
                            + [e^{-\phi t}(YC)^{\gamma - 1} - e^{-\phi t}C^{\gamma - 1}]dq  \\
                        &= [-\phi + (\gamma - 1)(\mu_c - \lambda k)
                            + \frac{1}{2}(\gamma - 1)(\gamma - 2) \sigma_c^2]Mdt
                            + (\gamma - 1)\sigma_c M dZ_c + (Y^{\gamma - 1} - 1)Mdq
            \end{aligned}
            \label{eq:prob1-b}
        \end{equation}
        \item   %% c.
        Since $\E \left[ \frac{dM}{M} \right] = -rdt$, the following equation holds.
        \begin{equation}
            \begin{aligned}
                r   &= -\E \left[ \frac{dM}{M} \right]  \\
                    &= \phi - (\gamma - 1)(\mu_c - \lambda k)
                        - \frac{1}{2} (\gamma - 1)(\gamma - 2) \sigma_c^2
                        - \lambda \E[e^{(\gamma - 1) \log Y} - 1]   \\
                    &= \phi - (\gamma - 1)(\mu_c - \lambda k)
                        - \frac{1}{2} (\gamma - 1)(\gamma - 2) \sigma_c^2
                        - \lambda (e^{(\gamma - 1) \alpha + \frac{1}{2} (\gamma - 1)^2 \delta^2} - 1)
            \end{aligned}
            \label{eq:prob1-c}
        \end{equation}
        Since $\mu_c , k, \lambda$ are constant, instantaneous risk free rate
        is constant.
        \item   %% d.
        Since an asset price is equal to sum of discounted future payoffs,
        assuming some regularity conditions hold, $S_t$ is represented as follows.
        \begin{equation}
            \begin{aligned}
            S_t     &= \E_t\left[ \int_t^\infty \frac{M_s}{M_t} D_s ds \right]  \\
                    &= \E_t\left[ \int_t^\infty \frac{e^{-\phi s} C_s^{\gamma - 1}}{e^{-\phi t} C_t^{\gamma - 1}} D_s ds \right] \\
            \Rightarrow \frac{S_t}{D_t} &= \E_t \left[ \int_t^\infty e^{-\phi(s - t)}
            \left( \frac{C_s}{C_t} \right)^{\gamma - 1} \left( \frac{D_s}{D_t} \right) ds \right]   \\
                    &= \E_t \left[ \int_t^\infty e^{-\phi(s - t)
                    + (\gamma - 1)\log(C_s / C_t) + \log(D_s / D_t)} ds \right] \\
                    &= \int_t^\infty \E_t \left[ e^{-\phi(s - t)
                    + (\gamma - 1)\log(C_s / C_t) + \log(D_s / D_t)} ds \right]
            \end{aligned}
            \label{eq:prob1-d-represent s}
        \end{equation}
        Considering the process of $C_t$, $\E_t[e^{(\gamma - 1) \log(C_s/C_t)}]$ is calculated
        as follows.
        \begin{equation}
            \begin{aligned}
                \E_t[e^{(\gamma - 1) \log(C_s/C_t)}] &= \E_t \left[
                e^{(\gamma - 1)(\mu_c - \frac{1}{2}\sigma_c^2 - \lambda k)(s - t)
                + \frac{1}{2} (\gamma - 1)^2 \sigma_c^2 (s - t) + (\gamma - 1) \log y(s, t)}
                \right] \\
                y(s, t) &= \prod_{i = s}^t Y_i
            \end{aligned}
            \label{eq:prob1-d-expected C}
        \end{equation}
        Since $\log y(s, t) = \sum_{i = s}^{t} \log Y_i$, and $\log Y_i$'s are independently
        and identically distributed as $N(\alpha, \delta^2)$, $\log(C_s/C_t)$ is also
        normally distributed, and its expected value from equation
        (\ref{eq:prob1-d-expected C}) is calculated as follows.
        \begin{equation}
            \E_t[e^{(\gamma - 1) \log(C_s/C_t)}] =
            e^{(\gamma - 1)(\mu_c - \frac{1}{2}\sigma_c^2 - \lambda k)(s - t)
            + \frac{1}{2} (\gamma - 1)^2 \sigma_c^2 (s - t)
            + (\gamma - 1) \alpha (s - t) + \frac{1}{2} (\gamma - 1)^2 \delta^2 (s - t)}
            \label{eq:prob1-d-expected logC}
        \end{equation}
        Applying the result from equation (\ref{eq:prob1-d-expected logC})
        and considering the correlation between $z_d$ and $z_c$ is $\rho$,
        $\frac{S_t}{D_t}$ from equation (\ref{eq:prob1-d-represent s})
        is solved as follows.
        \begin{equation}
            \begin{aligned}
                \frac{S_t}{D_t} &=
                \int_t^\infty {e^{-(s - t)[\phi + (1 - \gamma)(\mu_c - \frac{1}{2}\sigma_c^2 - \lambda k)
                + \frac{1}{2}(1 - \gamma)^2 \sigma_c^2
                + (1 - \gamma)\alpha
                + \frac{1}{2} (1 - \gamma)^2 \delta^2
                - \mu_d + (1 - \gamma) \rho \sigma_c \sigma_d]}} ds \\
                &=
                \left. -\frac{1}{A} e^{-(s - t)A} \right \rvert_{t}^{\infty}
                = \frac{1}{A}   \\
                A &= \phi + (1 - \gamma)(\mu_c - \frac{1}{2}\sigma_c^2 - \lambda k)
                + \frac{1}{2}(1 - \gamma)^2 \sigma_c^2
                + (1 - \gamma)\alpha
                + \frac{1}{2} (1 - \gamma)^2 \delta^2
                - \mu_d + (1 - \gamma) \rho \sigma_c \sigma_d
            \end{aligned}
            \label{eq:prob1-d-solution}
        \end{equation}
    \end{enumerate}
\end{homeworkProblem}

%----------------------------------------------------------------------------------------
%	PROBLEM 2
%----------------------------------------------------------------------------------------

\begin{homeworkProblem}
    \begin{enumerate}[a.]
        \item   %% a.
        Considering the process of risky asset price, intertemporal budget
        constraint is derived as follows.
        \begin{equation}
            \begin{aligned}
                dW  &= \omega_t \frac{dS}{S} + (1 - \omega_t) rdt - C_tdt   \\
                    &= (\omega_t(\mu - \lambda k - r)W_t + rW_t - C_t)dt + \sigma \omega_t W_t dz
                        + \omega_t W_t (Y_t - 1)dq
            \end{aligned}
            \label{eq:prob2-a}
        \end{equation}
        \item   %% b.
        Investors maximize $\E_0[\int_0^T e^{-\phi t}u(C_t)dt]$, subject to
        the equation (\ref{eq:prob2-a}). \\
        Let $J(W_t, t) = \max_{C_t, \omega_t} \E_t[\int_t^T e^{-\phi s}u(C_s)ds]$. Then the
        following equation follows.
        \begin{equation}
            \begin{aligned}
                J(W_t, t) &= \max_{C_t, \omega_t} \E_t \left[ \int_t^{t + \Delta t} e^{-\phi s}u(C_s)ds
                                    + J(W_{t + \Delta t}, t + \Delta t) \right]    \\
                        &= \max_{C_t, \omega_t} \E_t [ u(C_t)\Delta t
                        + J(W_t, t) + J_t \Delta t + J_W(\omega_t(\mu - \lambda k - r)W_t + rW_t - C_t)\Delta t  \\
                        &+ \frac{1}{2} \omega_t^2 W_t^2 \sigma^2 J_{WW}\Delta t
                        + (J(1 + \omega_t W_t (Y_t - 1), t) - J(W_t, t))dq ],
                        ~~~~ (t \in [0, T])
            \end{aligned}
            \label{eq:prob2-b-1}
        \end{equation}
        Letting $\Delta t \to 0$, equation (\ref{eq:prob2-b-1}) becomes
        equation (\ref{eq:prob2-b-2}), and it is the Bellman equation.
        \begin{equation}
            \begin{aligned}
                0   &= \max_{C, \omega} [u(C_t) + L(J)]    \\
                L(J) &=
                J_t + J_W(\omega_t W_t (\mu - \lambda k - r) + rW_t - C_t)
                 + \frac{1}{2} \omega_t^2 W_t^2 \sigma^2 J_{WW}
                 + \lambda \E_t[J(1 + \omega_t W_t (Y_t - 1), t) - J(W_t, t)]
            \end{aligned}
            \label{eq:prob2-b-2}
        \end{equation}
        \item   %% c.
        Applying first order condition to equation (\ref{eq:prob2-b-2}),
        the following equation holds.
        \begin{equation}
            \begin{aligned}
                &u_C     = J_W  \\
                &J_W W_t(\mu - \lambda k - r) + \omega_t W_t^2 \sigma^2 J_{WW} +
                \lambda \E_t[W_t (Y_t - 1) J_W(1 + \omega_t W_t (Y_t - 1), t)] = 0
            \end{aligned}
            \label{eq:prob2-c}
        \end{equation}
    \end{enumerate}
\end{homeworkProblem}
\end{document}
